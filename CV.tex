\documentclass[10pt]{article}

% DOCUMENT LAYOUT
%\usepackage{fullpage}
%\usepackage[cm]{fullpage}
%\usepackage[letterpaper, top=0.5in, bottom=0.5in, left=0.5in, right=0.5in]{geometry} 
%\usepackage[letterpaper]{geometry} 
%\geometry{textwidth=5.7in, textheight=9.0in, marginparsep=7pt, marginparwidth=.6in}
%\geometry{textwidth=6.0in, textheight=9.8in, marginparsep=7pt, marginparwidth=.8in}
%\geometry{textwidth=6in, textheight=9.8in, marginparsep=7pt, marginparwidth=.8in}
\setlength\parindent{0in}

%\addtolength{\oddsidemargin}{-.500in}
%\addtolength{\evensidemargin}{-.500in}
%\addtolength{\textwidth}{2.00in}

%\addtolength{\topmargin}{-.875in}
%\addtolength{\textheight}{1.75in}

% Dynamically control baselinestretch
\usepackage{setspace}

\addtolength{\topmargin}{-1.125in}
\addtolength{\textheight}{2.0in}

\addtolength{\oddsidemargin}{-.375in}
\addtolength{\evensidemargin}{-.375in}
\addtolength{\textwidth}{1.75in}
\addtolength{\marginparsep}{-10pt}
\addtolength{\marginparwidth}{+15pt}

\usepackage{wrapfig}

\usepackage{chngpage}
\usepackage[export]{adjustbox}

\usepackage{sidecap}
\usepackage[abs]{overpic}
\usepackage{wrapfig}

% SPECIAL FORMATTING
% ---- CUSTOM AMPERSAND
\newcommand{\amper}{{\selectfont\itshape\&}}
% ---- MARGIN YEARS
%\newcommand{\years}[1]{\marginpar{\scriptsize #1}}
\newcommand{\years}[1]{\marginpar{\quad \small #1}}
% ---- TALK COUNTER
\newcounter{talknumber}
\setcounter{talknumber}{1}
% ---- ARTICLE COUNTER
\newcounter{articlenumber}
\setcounter{articlenumber}{1}
%\newcommand{\article}{\noindent\marginpar{\scriptsize \arabic{articlenumber}}\addtocounter{articlenumber}{1}}
\newcommand{\article}[2]{
\noindent\marginpar{
  \parbox[c]{0.25in}{\scriptsize \arabic{articlenumber}} 
  \parbox[c]{0.25in}{\scriptsize \href{#2}{#1}}
}\addtocounter{articlenumber}{1}}
% PAPER DESCRIPTION AND THUMBNAIL IMAGE
\newcommand{\desc}[2]{
\noindent\marginpar{
  \parbox[c]{0.50in}{{\includegraphics[width=0.9in]{thumbnails/{#1}}}}
}{\footnotesize\emph{#2}}}
% NEW ARTICLE STYLE WITH IMAGE AND NO NUMBERS
% #1 : thumbnail filename
% #2 : author list
% #3 : article title
% #4 : journal reference
% #5 : DOI, bioRxiv, or ar$\chi$v
% #6 : DOI, bioRxiv, or arXiv link
% #7 : blurb / description
\newcommand{\newarticle}[7]{
\begin{adjustwidth}{-1in}{-1in}  
\begin{tabular}{p{0.9in}p{7in}}
\parbox[c]{0.9in}{\includegraphics[width=0.9in]{thumbnails/#1}} & \parbox[c]{6in}{\setstretch{0.9} {\scriptsize {#2}} \\ {\bf #3}  \\ {\small #4} $\cdot$ \href{#6}{#5} \\ {\footnotesize\emph {#7}}}
\end{tabular}
\end{adjustwidth}
\vspace{0.2in}
}

\newcommand{\newsoftware}[3]{
\begin{adjustwidth}{-1in}{-1in}  
\begin{tabular}{p{0.9in}p{7in}}
\parbox[c]{0.9in}{} & \parbox[c]{6in}{\setstretch{0.9} {\scriptsize {#1}} \\ {\bf #2}  \\ {\footnotesize\emph {#3}}}
\end{tabular}
\end{adjustwidth}
\vspace{0.2in}
}
% ---- TALKS
\newcommand{\talk}[2]{
\noindent\marginpar{
   \scriptsize \scshape
  #2 \\ #1
}}
\newcommand{\numberedtalk}[2]{
\noindent\marginpar{
  \parbox[c]{0.25in}{\scriptsize \arabic{talknumber}} 
% \parbox[c]{0.5in}{\scriptsize {#2}}
  \parbox[c]{0.5in}{\scriptsize \scshape {#1}}
}\addtocounter{talknumber}{1}}

\newcommand{\newtalk}[3]{
\noindent\marginpar{
   \scriptsize \scshape
  #2 \\ #1
}\spaceTalk\\}

% SPACING
\newcommand{\spaceEd}{\vspace{1ex}}
\newcommand{\spaceRes}{\vspace{1ex}}
\newcommand{\spaceFell}{\vspace{0.5ex}}
\newcommand{\spacePub}{\vspace{0.5ex}}
\newcommand{\spaceTalk}{\vspace{1ex}}
\newcommand{\spaceSoc}{\vspace{0.5ex}}
\newcommand{\spacePost}{\vspace{0.5ex}}

% Modify section spacing
\usepackage{titlesec}
\titlespacing*{\section}{0pt}{0.2\baselineskip}{0.2\baselineskip}
\titlespacing*{\subsection}{0pt}{0.2\baselineskip}{0.2\baselineskip}
\titlespacing*{\subsubsection}{0pt}{0.2\baselineskip}{0.2\baselineskip}

% GRAPHICS
\usepackage{graphicx}

% HEADINGS
\usepackage{sectsty} 
\usepackage[normalem]{ulem} 

\usepackage{fancyhdr}
\renewcommand{\headrulewidth}{0pt}
\pagestyle{fancy}
\fancyhf{}
%\fancyfoot[LE,CO]{{\small\textsc{John D. Chodera - Sloan Kettering Institute - \thepage}}}%
\rfoot{\small\textsc{Henrique M. Cezar - University of S\~ao Paulo - \thepage}}

% Clean up font spacing with micro type
\usepackage{microtype}

% Use a clean, modern-looking font
\usepackage[default]{lato}
% \usepackage{libertine}
\usepackage{FiraSans}
\usepackage[T1]{fontenc}

% Change the appearance of section and subsection headings
\sectionfont{\mdseries\large\underline} 
\subsectionfont{\mdseries\scshape\normalsize} 
\subsubsectionfont{\bfseries\upshape\normalsize} 

% Control tightness of spacing
%\renewcommand{\baselinestretch}{1.00}

% PDF SETUP
% ---- FILL IN HERE THE DOC TITLE AND AUTHOR
\usepackage[bookmarks, colorlinks, breaklinks, pdftitle={Henrique M. Cezar - vita},pdfauthor={Henrique M. Cezar}]{hyperref}  
\hypersetup{linkcolor=blue,citecolor=blue,filecolor=black,urlcolor=blue} 

% Change the typewriter font to something nicer.
\usepackage[ttdefault]{sourcecodepro}

% DOCUMENT
\begin{document}
\reversemarginpar
% FILL IN NAME HERE
{\fontseries{eb}\selectfont \LARGE Henrique Musseli Cezar}\\[0.5cm]

% FILL IN AUTHOR INFORMATION HERE
\begin{minipage}[t]{2.5in}
%\Large CONTACT
%%\includegraphics[width=0.7in,valign=c]{images/john_chodera_sm.pdf}
\hspace{35pt}
\includegraphics[width=1.3in,valign=c]{images/henrique.jpg}
\end{minipage}
%\quad
\begin{minipage}[t]{3in}
\begin{tabular}{rl}
{\bf email} & \href{mailto:henrique.cezar@usp.br}{\href{mailto:henrique.cezar@usp.br}{henrique.cezar@usp.br}}\\[0.05in]
{\bf github} & \href{https://github.com/hmcezar}{https://github.com/hmcezar}\\[0.05in]
{\bf orcid} & \href{https://orcid.org/0000-0002-7553-0482}{0000-0002-7553-0482}\\[0.05in]
{\bf tel} & +55 11 3091-6790\\[0.05in]
{\bf mobile} & +55 11 96722-8992\\[0.05in]
{\bf post} & 
\parbox[t]{3.0in}{
1371 Rua do Mat\~ao, Ala 1, Room 2070\\
S\~ao Paulo, SP, 05508-090}
\end{tabular}
\end{minipage}

\vspace{15pt} 

\noindent\fbox{%
    \parbox{\textwidth}{ 
    {Ph.D in Physics, works with computational physics/chemistry and developing scientific software. My research interests are in the development of new methods for molecular simulation and the use of molecular simulation and quantum mechanics calculations to solve problems in physical chemistry, biochemistry and materials science.}
    }
}

% \setstretch{1.00} 

\section*{Education and positions}

\noindent\years{2018--2019}\textbf{Postdoctoral researcher, Institute of Physics, University of S\~ao Paulo}\\
Development and applications of Configurational Bias Monte Carlo Method in dyes of interest to organic solar cells \\
CAPES Fellowship - Supervisor: Dr. Kaline R. Coutinho \\
\noindent\years{2015--2018}\textbf{Ph.D. in Physics, Institute of Physics, University of S\~ao Paulo}\\
Implementation and Development of Efficient Algorithm for Intramolecular Deformation with the Monte Carlo Method \\
CNPq Ph.D. Fellowship - Advisor: Dr. Kaline R. Coutinho \\
\noindent\years{2013--2015}\textbf{Master's in Computational Physics, S\~ao Carlos Institute of Physics, University of S\~ao Paulo}\\
Implementation of the Parallel Tempering Monte Carlo Method to the Study of Thermodynamic Properties of Nanoclusters \\
FAPESP Master Fellowship - Advisor: Dr. Juarez L. F. Da Silva \\
\noindent\years{2009--2012}\textbf{B.S. in Computational Physics, S\~ao Carlos Institute of Physics, University of S\~ao Paulo}\\
Undergraduate research entitled: Fluid Dynamics in Porous Media with Lattice Boltzmann \\ 
CNPq Scientific Initiation Fellowship - Advisor: Dr. Leonardo P. Maia

\section*{Complementary Education}
\noindent\years{2018}{\textbf{Short Course:} I GPU Computing Workshop (Duration: 11h). University of S\~ao Paulo.} \\
\noindent\years{2016}{\textbf{Short Course:} 4th Workshop HPC - USP/Rice (Duration: 8h). University of S\~ao Paulo}. \\
\noindent\years{2015}{\textbf{Coursera Online Course:} Algorithms: Design and Analysis, Part 1. (Duration: 36h). Stanford University.} \\
\noindent\years{2006--2008}{\textbf{Technical Course:} Informatics. Col\'egio Divino Salvador.}

% \section*{Fellowships}

% \noindent\years{2015--}CNPq Ph.D. Fellowship\\
% \noindent\years{2013--2015}FAPESP Master Fellowship\\
% \noindent\years{2012--2012}CNPq Scientific Initiation Fellowship

\section*{Awards}
\noindent\years{2019}{\textbf{3rd place} in the best poster award at the XX Simp\'osio Brasileiro de Qu\'imica Te\'orica.} \\
\noindent\years{2017}{\textbf{Best poster} award at the XIX Simp\'osio Brasileiro de Qu\'imica Te\'orica.} \\
\noindent\years{2015}{\textbf{Honorable mention} at the best poster award of the II Workshop on Biomolecular Theory-Experiment Interplay.}

\section*{Teaching Experience}
\noindent\years{2019}{\textbf{F\'isica II}: Teaching the second semester physics class at the Escola Polit\'ecnica of the University of S\~ao Paulo.} \\
\noindent\years{2019}{\textbf{F\'isica I}: Teaching the first semester physics class at the Escola Polit\'ecnica of the University of S\~ao Paulo.} \\
\noindent\years{2017}{\textbf{F\'isica Moderna I}: Teaching assistant for the introductory modern physics class of the Physics course of University of S\~ao Paulo, under the supervision of Prof. Dr. Sylvio Canuto.}


\section*{Research overview}
My research is mainly focused on the development of algorithms and computer code to efficiently sample properties of systems with intricate potential energy surfaces.
Specifically, I have experience implementing and using Monte Carlo methods and extended ensemble methods such as replica exchange.
During my Master I have implemented a parallel tempering Monte Carlo algorithm with several options in GOTNano, and have used this same software, together with FHI-aims, to investigate the properties of nanoclusters.
In my PhD I developed a Configurational Bias Monte Carlo method for the efficient simulation of flexible molecules in solution, showing that the method may perform better than standard molecular dynamics for the sampling of systems with high energy barriers between conformers.

Even though most of my research has been in developing and implementing computational methods, I also have interest in combining different techniques to solve problems in physical chemistry and materials science.
In my career I have used different methods such as molecular dynamics for the simulation of molecular systems, global optimization algorithms to search for the putative global minimum configurations, clustering algorithms to analyze data from simulation and density functional theory to investigate the electronic structure and absorption spectra of molecules and transition metal nanoclusters.
All these methods form a useful toolset that can be used to investigate matter in the solid, liquid and gas phases.

\section*{Computational Skills}
I am proficient with the programming languages (from more to less proficient): Fortran (legacy and modern), C, Python, C++ and Bash. I have experience with OpenMP parallelization, code profiling and optimization, and also have knowledge in message passaging parallelization with MPI and GPU and shared memory parallelization with OpenACC. I know how to use the Git and SVN version control systems.

Concerning scientific software, I am familiar with DICE, GOTNano and GROMACS for molecular simulation. For electronic structure calculations I have used Gaussian and FHI-AIMS.

% \setstretch{1.00} 

% \eject

%%%%%%%%%%%%


\section*{Software development}
Apart from several scripts and smaller software, my major contributions are developing:\\

\newsoftware{\textbf{Cezar HM}, Canuto S, Coutinho K}{DICE}{A Monte Carlo code for the molecular simulation of liquids and gases using classical force fields. My main contribution is the coding and development of a Configurational Bias Monte Carlo method used to sample the internal degrees of freedom of the molecules. I have also implemented shared memory parallelization and improved the overall performance. The code is written in Fortran and has OpenMP parallelization.}

\newsoftware{Rondina GG, \textbf{Cezar HM}, Da Silva JLF}{GOTNano}{A code for global optimization and sampling of thermodynamic properties of nanoclusters. My main contribution is the implementation of the Parallel Tempering Monte Carlo algorithm used to sample the thermodynamic properties. The code is written in C and has OpenMP parallelization.}

\newsoftware{\textbf{Cezar HM}}{Clustering Trajectory}{\href{https://github.com/hmcezar/clustering-traj}{https://github.com/hmcezar/clustering-traj} \\ A parallel Python script that performs clustering over an atomistic molecular simulation trajectory, searching for the minimum RMSD between each structure and using one of the many clustering algorithms available in the machine learning library sklearn.}

\newsoftware{\textbf{Cezar HM}}{DICEtools}{\href{https://github.com/hmcezar/dicetools}{https://github.com/hmcezar/dicetools} \\ A package containing several scripts used to prepare DICE inputs and analyze simulation data.}

%%%%%%%%%%%%%%%%%%%%%%%%%%%%%%%%%%%%%%%%%%%%%%%%%%%%%%%%%%%%%%%%%%%%%%%%%%%%%%%%
% PUBLICATIONS
%%%%%%%%%%%%%%%%%%%%%%%%%%%%%%%%%%%%%%%%%%%%%%%%%%%%%%%%%%%%%%%%%%%%%%%%%%%%%%%%

\section*{Publications}

%%%%%%%%%%%%

\textit{h-index: 3, citations: 17 (16 Nov 2019)} \\
\href{https://publons.com/researcher/2003548/}{Publons}, 
\href{https://scholar.google.com.br/citations?hl=pt-BR&user=LtBk3gEAAAAJ}{Google Scholar} 

% \subsection*{In Press}

%\eject

%%%%%%%%%%%%

\subsection*{Published}

\newarticle{cbmc_mox.pdf}{\textbf{Cezar HM}, Canuto S, Coutinho K}{Solvent effect on the syn/anti conformational stability: A comparison between conformational bias Monte Carlo and molecular dynamics methods}{\emph{International Journal of Quantum Chemistry} 119:e25688, 2019}{DOI}{http://dx.doi.org/10.1002/qua.25688}{We compare the Configurational Bias Monte Carlo (CBMC) and molecular dynamics (MD) sampling of the mesityloxide molecule in gas phase and solution, showing that that while CBMC performs an ergodic sampling, MD has troubles overcoming the 10 kcal/mol energy barrier between the \textit{syn} and \textit{anti} conformations.}

\newarticle{similarity_lj.pdf}{\textbf{Cezar HM}, Rondina GG, Da Silva JLF}{Parallel Tempering Monte Carlo Combined with Clustering Euclidean Metrics Analysis to Study the Thermodynamic Stability of Lennard-Jones Nanoclusters}{\emph{Journal of Chemical Physics} 146:064114, 2017}{DOI}{http://dx.doi.org/10.1063/1.4975601}{We use a clustering algorithm to analyze trajectories from Parallel Tempering Monte Carlo simulations of Lennard-Jones nanoclusters, identifying the phase changes and most most frequent structures with great accuracy using a very simple approach.}

\newarticle{alloys_douglas.pdf}{De Souza DG, \textbf{Cezar HM}, Rondina GG, De Oliveira MF, Da Silva JLF}{A Basin-hopping Monte Carlo Investigation of the Structural and Energetic Properties of 55- and 561-atom Bimetallic Nanoclusters: the Examples of the ZrCu, ZrAl, and CuAl Systems}{\emph{Journal of Physics: Condensed Matter} 28:175302, 2016}{DOI}{http://dx.doi.org/10.1088/0953-8984/28/17/175302}{We report a basin-hopping Monte Carlo investigation of the structural and energetic properties of bimetallic ZrCu, ZrAl, and CuAl nanoclusters with 55 and 561 atoms, showing how the atoms of the different species are distributed in the structure, and a trend of more spherical structures at 50-50 compositions.}

\subsection*{Accepted for publication}
\newarticle{ptmc_pt_alloys.png}{\textbf{Cezar HM}, Rondina GG, Da Silva JLF}{Thermodynamic Properties of 55-Atom Pt-based Nanoalloys: Phase Changes and Structural Effects on the Electronic Properties}{\emph{Journal of Chemical Physics}}{}{}{We investigate the effects of temperature on the structure of Pt-based nanoalloys, performing Parallel Tempering Monte Carlo simulations using many-body potentials and DFT calculations to see how the structural changes affect the density of states.}

\subsection*{In preparation}
I have 3 manuscripts in different steps of preparation, of which I highlight 2 that are under the review of the coauthors and should be submitted in the next weeks.
\\

\newarticle{cbmc_dice.png}{\textbf{Cezar HM}, Canuto S, Coutinho K}{DICE v3.0: A Monte Carlo code for molecular simulation including Configuration Bias Monte Carlo method}{}{}{}{We describe DICE, a software that uses Monte Carlo methods to perform molecular simulation, focusing on solute-solvent systems. We introduce our implementation of the Configurational Bias Monte Carlo method that includes new methodological developments.}

\newarticle{dha-vhf.png}{Cardenuto MH, \textbf{Cezar HM}, Mikkelsen KV, Sauer SPA, Coutinho K, Canuto S.}{A QM/MM study of the absorption spectra of DHA/VHF photochromic switches in acetonitrile solution}{}{}{}{The absorption spectra of photoswichtes composed of two units of photochromic molecules, namely, dihydroazulene (DHA)/vinylheptafulvene(VHF) is studied including explicit solvation and considering the rotamers that are exhibited at room temperature.}

\section*{Invited and contributed talks}
\talk{Sep 2019}{S\~ao Carlos, SP}\textbf{DICEtools and Clustering-Traj: From simulation setup to data analysis} \\ III New Energies Innovation Center - Computational Material Science Division Workshop\\[0.05in]

\talk{Feb 2019}{S\~ao Carlos, SP}\textbf{Monte Carlo Methods} \\ II New Energies Innovation Center - Computational Material Science Division Workshop (Invited talk)\\[0.05in]

\talk{Sep 2018}{Bras\'ilia, DF}\textbf{Comparison Between Configurational Bias Monte Carlo and Molecular Dynamics Sampling for Sampling Conformational Stability with High Energy Barriers} \\ VII SeedMol  \\[0.05in]

\talk{Aug 2017}{Livorno, Italy}\textbf{Configurational Bias Monte Carlo of Molecules in Solvent and Comparison with Molecular Dynamics} \\ Coding Solvation Workshop \\[0.05in]

\talk{May 2017}{Recife, PE}\textbf{Comparsion of the sampling of Molecular Dynamics and Configurational Bias Monte Carlo for 1,2- dichloroethane and octane.} \\ III Advanced School on Biomolecular Simulation \\[0.05in]

\talk{Mar 2017}{Rio de Janeiro, RJ}\textbf{How Indiana Jones is actually a bad explorer when it comes to molecules and how I can do better} \\ Finals of My Thesis in 180 seconds (Swissnex
Brazil) \\[0.05in]

\talk{Sep 2014}{Jo\~ao Pessoa, PB}\textbf{Parallel Tempering Monte Carlo Investigation of Phase-Changes in Nanoclusters} \\ XIII Encontro da SBPMat \\[0.05in]



\section*{Poster presentation}
\talk{Nov 2019}{Jo\~ao Pessoa, PB}\textbf{Conformational effects on the solvatochromism of mesityl oxide: the importance of an ergodic sampling} \\ XX SBQT \\[0.05in]

\talk{Feb 2019}{S\~ao Carlos, SP}\textbf{Application of Configurational Bias Monte Carlo method for the simulation of dyes and porous nanoparticles} \\ II New Energies Innovation Center - Computational Material Science Division Workshop \\[0.05in]

\talk{Oct 2018}{S\~ao Carlos, SP}\textbf{Implementation of Configurational Bias Monte Carlo method for the simulation of flexible molecules in solution} \\ I New Energies Innovation Center - Computational Material Science Division Workshop \\[0.05in]

\talk{Nov 2017}{\'Aguas de Lind\'oia, SP}\textbf{Implementation of Configurational Bias Monte Carlo Method to Sample Flexible Solute in Solvent Media} \\ XIX SBQT \\[0.05in]

\talk{Aug 2017}{Munich, Germany}\textbf{Configurational Bias Monte Carlo method to sample molecular flexibility: The case of octane and 1,2-dichloroethane} \\ WATOC 2017 \\[0.05in]

\talk{Aug 2015}{Maresias, SP}\textbf{Parallel Tempering Monte Carlo applied to the study of transition-metal nanoclusters} \\ II Workshop on Biomolecular Theory-Experiment Interplay \\[0.05in]

\talk{Aug 2013}{Trieste, Italy}\textbf{Thermodynamic properties of nanoclusters: An investigation with Parallel Tempering Monte Carlo} \\ Density Functional Theory and Beyond: Computational Materials Science for Real Materials \\[0.05in]

\talk{Oct 2012}{S\~ao Paulo, SP}\textbf{Estudo de din\^amica de fluidos em meios porosos com m\'etodos de Boltzmann na rede} \\ XX SIICUSP \\

\section*{Conferences organized}
\talk{May 2017}{Recife, PE}\textbf{III Advanced School on Biomolecular Simulation} \\ Member of the local committee.

% \eject

\vfill{}
% \hrulefill
% credits to Chodera's template
% \begin{center}
% {\scriptsize
% The \LaTeX{} template for this CV is forked from \href{https://github.com/jchodera/latex-cv}{https://github.com/jchodera/latex-cv}}
% \end{center}

\end{document}
