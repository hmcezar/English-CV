\documentclass[10pt]{article}

% DOCUMENT LAYOUT
%\usepackage{fullpage}
%\usepackage[cm]{fullpage}
%\usepackage[letterpaper, top=0.5in, bottom=0.5in, left=0.5in, right=0.5in]{geometry} 
%\usepackage[letterpaper]{geometry} 
%\geometry{textwidth=5.7in, textheight=9.0in, marginparsep=7pt, marginparwidth=.6in}
%\geometry{textwidth=6.0in, textheight=9.8in, marginparsep=7pt, marginparwidth=.8in}
%\geometry{textwidth=6in, textheight=9.8in, marginparsep=7pt, marginparwidth=.8in}
\setlength\parindent{0in}

%\addtolength{\oddsidemargin}{-.500in}
%\addtolength{\evensidemargin}{-.500in}
%\addtolength{\textwidth}{2.00in}

%\addtolength{\topmargin}{-.875in}
%\addtolength{\textheight}{1.75in}

% Dynamically control baselinestretch
\usepackage{setspace}

\addtolength{\topmargin}{-1.125in}
\addtolength{\textheight}{2.0in}

\addtolength{\oddsidemargin}{-.375in}
\addtolength{\evensidemargin}{-.375in}
\addtolength{\textwidth}{1.75in}
\addtolength{\marginparsep}{-10pt}
\addtolength{\marginparwidth}{+15pt}

\usepackage{wrapfig}

\usepackage{chngpage}
\usepackage[export]{adjustbox}

\usepackage{sidecap}
\usepackage[abs]{overpic}
\usepackage{wrapfig}

% SPECIAL FORMATTING
% ---- CUSTOM AMPERSAND
\newcommand{\amper}{{\selectfont\itshape\&}}
% ---- MARGIN YEARS
%\newcommand{\years}[1]{\marginpar{\scriptsize #1}}
\newcommand{\years}[1]{\marginpar{\quad \small #1}}
% ---- TALK COUNTER
\newcounter{talknumber}
\setcounter{talknumber}{1}
% ---- ARTICLE COUNTER
\newcounter{articlenumber}
\setcounter{articlenumber}{1}
%\newcommand{\article}{\noindent\marginpar{\scriptsize \arabic{articlenumber}}\addtocounter{articlenumber}{1}}
\newcommand{\article}[2]{
\noindent\marginpar{
  \parbox[c]{0.25in}{\scriptsize \arabic{articlenumber}} 
  \parbox[c]{0.25in}{\scriptsize \href{#2}{#1}}
}\addtocounter{articlenumber}{1}}
% PAPER DESCRIPTION AND THUMBNAIL IMAGE
\newcommand{\desc}[2]{
\noindent\marginpar{
  \parbox[c]{0.50in}{{\includegraphics[width=0.9in]{thumbnails/{#1}}}}
}{\footnotesize\emph{#2}}}
% NEW ARTICLE STYLE WITH IMAGE AND NO NUMBERS
% #1 : thumbnail filename
% #2 : author list
% #3 : article title
% #4 : journal reference
% #5 : DOI, bioRxiv, or ar$\chi$v
% #6 : DOI, bioRxiv, or arXiv link
% #7 : blurb / description
\newcommand{\newarticle}[7]{
\begin{adjustwidth}{-1in}{-1in}  
\begin{tabular}{p{0.9in}p{7in}}
\parbox[c]{0.9in}{\includegraphics[width=0.9in]{thumbnails/{#1}}} & \parbox[c]{6in}{\setstretch{0.9} {\scriptsize {#2}} \\ {\bf #3}  \\ {\small #4} $\cdot$ \href{#6}{#5} \\ {\footnotesize\emph {#7}}}
\end{tabular}
\end{adjustwidth}
\vspace{0.2in}
}

\newcommand{\newsoftware}[3]{
\begin{adjustwidth}{-1in}{-1in}  
\begin{tabular}{p{0.9in}p{7in}}
\parbox[c]{0.9in}{} & \parbox[c]{6in}{\setstretch{0.9} {\scriptsize {#1}} \\ {\bf #2}  \\ {\footnotesize\emph {#3}}}
\end{tabular}
\end{adjustwidth}
\vspace{0.2in}
}
% ---- TALKS
\newcommand{\talk}[2]{
\noindent\marginpar{
   \scriptsize \scshape
  #2 \\ #1
}}
\newcommand{\numberedtalk}[2]{
\noindent\marginpar{
  \parbox[c]{0.25in}{\scriptsize \arabic{talknumber}} 
% \parbox[c]{0.5in}{\scriptsize {#2}}
  \parbox[c]{0.5in}{\scriptsize \scshape {#1}}
}\addtocounter{talknumber}{1}}

\newcommand{\newtalk}[3]{
\noindent\marginpar{
   \scriptsize \scshape
  #2 \\ #1
}\spaceTalk\\}

% SPACING
\newcommand{\spaceEd}{\vspace{1ex}}
\newcommand{\spaceRes}{\vspace{1ex}}
\newcommand{\spaceFell}{\vspace{0.5ex}}
\newcommand{\spacePub}{\vspace{0.5ex}}
\newcommand{\spaceTalk}{\vspace{1ex}}
\newcommand{\spaceSoc}{\vspace{0.5ex}}
\newcommand{\spacePost}{\vspace{0.5ex}}

% Modify section spacing
\usepackage{titlesec}
\titlespacing*{\section}{0pt}{0.2\baselineskip}{0.2\baselineskip}
\titlespacing*{\subsection}{0pt}{0.2\baselineskip}{0.2\baselineskip}
\titlespacing*{\subsubsection}{0pt}{0.2\baselineskip}{0.2\baselineskip}

% GRAPHICS
\usepackage{graphicx}

% HEADINGS
\usepackage{sectsty} 
\usepackage[normalem]{ulem} 

\usepackage{fancyhdr}
\renewcommand{\headrulewidth}{0pt}
\pagestyle{fancy}
\fancyhf{}
%\fancyfoot[LE,CO]{{\small\textsc{John D. Chodera - Sloan Kettering Institute - \thepage}}}%
\rfoot{\small\textsc{Henrique M. Cezar - University of S\~ao Paulo - \thepage}}

% Clean up font spacing with micro type
\usepackage{microtype}

% Use a clean, modern-looking font
\usepackage[default]{lato}
\usepackage[T1]{fontenc}

% Change the appearance of section and subsection headings
\sectionfont{\mdseries\large\underline} 
\subsectionfont{\mdseries\scshape\normalsize} 
\subsubsectionfont{\bfseries\upshape\normalsize} 

% Control tightness of spacing
%\renewcommand{\baselinestretch}{1.00}

% PDF SETUP
% ---- FILL IN HERE THE DOC TITLE AND AUTHOR
\usepackage[bookmarks, colorlinks, breaklinks, pdftitle={Henrique M. Cezar - vita},pdfauthor={Henrique M. Cezar}]{hyperref}  
\hypersetup{linkcolor=blue,citecolor=blue,filecolor=black,urlcolor=blue} 

% Change the typewriter font to something nicer.
\usepackage[ttdefault]{sourcecodepro}

% DOCUMENT
\begin{document}
\reversemarginpar
% FILL IN NAME HERE
{\fontseries{eb}\selectfont \LARGE Henrique Musseli Cezar}\\[0.5cm]

% FILL IN AUTHOR INFORMATION HERE
\begin{minipage}[t]{2.5in}
%\Large CONTACT
%%\includegraphics[width=0.7in,valign=c]{images/john_chodera_sm.pdf}
\hspace{35pt}
\includegraphics[width=1.3in,valign=c]{images/henrique.png}
\end{minipage}
%\quad
\begin{minipage}[t]{3in}
\begin{tabular}{rl}
{\bf email} & \href{mailto:henrique.cezar@usp.br}{\href{mailto:henrique.cezar@usp.br}{henrique.cezar@usp.br}}\\[0.05in]

{\bf tel} & +55 11 3091-6662\\[0.05in]
{\bf mobile} & +55 11 96722-8992\\[0.05in]
{\bf post} & 
\parbox[t]{3.0in}{
1371 Rua do Mat\~ao, Ala 1, Room 223\\
S\~ao Paulo, SP, 05508-090}
\end{tabular}
\end{minipage}

\vspace{15pt} 

\noindent\fbox{%
    \parbox{\textwidth}{ 
    {A Ph.D. candidate working with computational physics/chemistry and developing scientific software. My main research interest is the efficient sampling of molecular structure and thermodynamic properties.}
    }
}

\setstretch{1.05} 

\section*{Education}

\noindent\years{2015--}{\bfseries \textsc{Ph.D. Candidate} in Physics, Institute of Physics, University of S\~ao Paulo}\\
Implementation and Development of Efficient Algorithm for Intramolecular Deformation with the Monte Carlo Method \\
CNPq Ph.D. Fellowship - Advisor: Dr. Kaline R. Coutinho \\
\noindent\years{2013--2015}{\bfseries \textsc{Master's} in Computational Physics, S\~ao Carlos Institute of Physics, University of S\~ao Paulo}\\
Implementation of the Parallel Tempering Monte Carlo Method to the Study of Thermodynamic Properties of Nanoclusters \\
FAPESP Master Fellowship - Advisor: Dr. Juarez L. F. Da Silva \\
\noindent\years{2009--2012}{\bfseries \textsc{B.S.} in Computational Physics, S\~ao Carlos Institute of Physics, University of S\~ao Paulo}\\
Undergraduate Research with Dr. Leonardo P. Maia: Fluid Dynamics in Porous Media with Lattice Boltzmann - CNPq Scientific Initiation Fellowship

\section*{Complementary Education}
\noindent\years{2015}{\textsc{Coursera Online Course:} Algorithms: Design and Analysis, Part 1. (Duration: 36h). Stanford University}
\noindent\years{2006--2008}{\textsc{Technical Course} in Informatics, Col\'egio Divino Salvador}

% \section*{Fellowships}

% \noindent\years{2015--}CNPq Ph.D. Fellowship\\
% \noindent\years{2013--2015}FAPESP Master Fellowship\\
% \noindent\years{2012--2012}CNPq Scientific Initiation Fellowship

\section*{Computational Skills}
Proficient with programming languages (from more to less proficient): Fortran (legacy and modern), C, Python, C++ and Bash. I have experience with OpenMP parallelization, code profiling and optimization, and also have knowledge in MPI and Pthreads parallelization.

\section*{Research overview}
My research is mainly focused on the development of algorithms and computer code to efficiently sample properties of organic molecules and metallic nanoclusters with intricate potential energy surface and/or embedded in solvent.
Since molecular properties are often related to their structures, the sampling of molecular conformations is important for studies that goes from energy storage to the development of new drugs.
Specifically, I have experience with Monte Carlo methods and extended ensemble methods such as replica exchange.
I also have some experience with global optimization and clustering.
Recently, my work has been focused on the development of a Bias Monte Carlo method to sample internal degrees of freedom of organic molecules and small peptides in solution, which is the main subject of my Ph.D. thesis.
To achieve the objectives of my research, my work rely on high performance computing, and for this reason, I also have some interest in this field.

\setstretch{1.00} 

\eject

%%%%%%%%%%%%


\section*{Software development}
Apart from several scripts and smaller software, I am one of the main developers of two scientific softwares:\\

\newsoftware{{\bfseries Cezar HM}, Canuto S, Coutinho K}{Dice}{A Monte Carlo code for the molecular simulation of liquids and gases using classical force fields. My main contribution is the coding and development of a Configurational Bias Monte Carlo method used to sample the internal degrees of freedom of the molecules. The code is written in Fortran and has OpenMP parallelization.}

\newsoftware{Rondina GG, {\bfseries Cezar HM}, Da Silva JLF}{GOTNano}{A code for global optimization and sampling of thermodynamic properties of nanoclusters. My main contribution is the implementation of the Parallel Tempering Monte Carlo algorithm used to sample the thermodynamic properties. The code is written in C and has OpenMP parallelization.}


%%%%%%%%%%%%%%%%%%%%%%%%%%%%%%%%%%%%%%%%%%%%%%%%%%%%%%%%%%%%%%%%%%%%%%%%%%%%%%%%
% PUBLICATIONS
%%%%%%%%%%%%%%%%%%%%%%%%%%%%%%%%%%%%%%%%%%%%%%%%%%%%%%%%%%%%%%%%%%%%%%%%%%%%%%%%

\section*{Publications}

%%%%%%%%%%%%

\subsection*{In Press}

\newarticle{similarity_lj.pdf}{{\bfseries Cezar HM}, Rondina GG, Da Silva JLF}{Parallel Tempering Monte Carlo Combined with Clustering Euclidean Metrics Analysis to Study the Thermodynamic Stability of Lennard-Jones Nanoclusters}{\emph{Accepted for publication at Journal of Chemical Physics}}{}{}{We use a clustering algorithm to analyze trajectories from Parallel Tempering Monte Carlo simulations of Lennard-Jones nanoclusters, identifying the phase changes and most most frequent structures with great accuracy using a very simple approach.}

%\eject

%%%%%%%%%%%%

\subsection*{Published}

\newarticle{alloys_douglas.pdf}{De Souza DG, {\bfseries Cezar HM}, Rondina GG, De Oliveira MF, Da Silva JLF}{A Basin-hopping Monte Carlo Investigation of the Structural and Energetic Properties of 55- and 561-atom Bimetallic Nanoclusters: the Examples of the ZrCu, ZrAl, and CuAl Systems}{\emph{Journal of Physics: Condensed Matter} 28:175302, 2016}{DOI}{http://dx.doi.org/10.1088/0953-8984/28/17/175302}{We report a basin-hopping Monte Carlo investigation of the structural and energetic properties of bimetallic ZrCu, ZrAl, and CuAl nanoclusters with 55 and 561 atoms, showing how the atoms of the different species are distributed in the structure, and a trend of more spherical structures at 50-50 compositions.}

% \eject

\vfill{}
\hrulefill
% credits to Chodera's template
\begin{center}
{\scriptsize
The \LaTeX{} template for this CV is forked from \href{https://github.com/jchodera/latex-cv}{https://github.com/jchodera/latex-cv}}
\end{center}

\end{document}
